\documentclass[a4paper,11pt,twoside]{article}
\usepackage[utf8]{inputenc}	% Text coding
\usepackage[T1]{fontenc}
\usepackage{lmodern}
\usepackage[czech]{babel}
\usepackage{epsfig}
\usepackage{amsfonts,amsmath,amssymb}
\usepackage{graphicx}
\usepackage[unicode]{hyperref}
\usepackage{indentfirst}
\usepackage{fancyhdr}
\usepackage{xifthen}
\usepackage{amsthm,thmtools}
\usepackage{bold-extra}
\usepackage[dvipsnames]{xcolor}
\usepackage[subrefformat=simple,labelformat=simple]{subcaption} % Instead of subfigure
\usepackage{listings}
\usepackage{comment}
\usepackage{titlesec}
\usepackage{underscore}
\usepackage{makecell}       % Šířky čar v tabulkách

% Page size
\addtolength{\topmargin}{-1.5cm} %\addtolength{\textheight}{-10cm}
\addtolength{\textwidth}{4cm} \addtolength{\textheight}{4cm} % Width and height of the text
\addtolength{\voffset}{-0.5cm} % Top margin
\addtolength{\hoffset}{-2cm}
\setlength{\headheight}{15pt}

\DeclareMathOperator{\e}{e}

\def\vector#1{\boldsymbol{#1}}								% Vector
\renewcommand{\d}{\mathrm{d}}
\newcommand{\derivative}[3][]{\ifthenelse{\isempty{#1}}	    % Normal derivative
	{\frac{\d{#2}}{\d{#3}}}
	{\frac{\d^{#1}{#2}}{\d{#3}^{#1}}}
}
\newcommand{\im}{\mathrm{i}}

\def\makematrix#1{\begin{pmatrix}#1\end{pmatrix}}       % Matrix
\def\abs#1{\left|#1\right|}
\def\probability#1{\mathrm{Pr}\left[#1\right]}
\def\expectation#1{\mathrm{E}\left[#1\right]}
\def\dispersion#1{\sigma_{#1}^{2}}

\def\code#1{\textnormal{\texttt{#1}}}
\def\file#1{\textnormal{\textbf{\texttt{#1}}}}
\def\ghfile#1#2{\textnormal{\textbf{\texttt{\href{https://github.com/PavelStransky/PCInPhysics2021/blob/main/#1#2}{#2}}}}}

\def\abbreviation#1{\textnormal{\textsc{#1}}}

\begin{document}

\section*{Domácí úkol na 24.3.2022}
\subsection*{Hledání minima obecné funkce $D$ proměnných}
\begin{enumerate}
    \item Naprogramujte a odlaďte náhodnou procházku v $D$-rozměrném prostoru.
        Nezapomeňte, že krok do každého směru musí být stejně pravděpodobný, tj. pokud budeme opakovat náhodný krok dané délky $d$ z počátku souřadné soustavy, pokryjí koncové body kroků kouli o poloměru $d$.

    \item Pomocí $D$-rozměrné náhodné procházky nalezněte minimum funkce 4 proměnných
        \begin{align*}
            h(s,t,u,v)&=\frac{1}{2}\left(s^2+t^2+u^2+v^2\right)\\
                &\qquad-\frac{1}{4}\left[\left(s^2+t^2\right)\left(2-s^2-t^2-u^2-v^2\right)+\left(su-tv\right)^2\right]\\
                &\qquad+\frac{s}{4}\sqrt{2-s^2-t^2-u^2-v^2}.
        \end{align*}
        a minimum zobecněné Rosenbrockovy funkce
        \begin{equation*}
            r(\vector{x})=\sum_{j=1}^{D-1}\left[\left(a-x_{i}\right)^{2}+b\left(x_{i+1}-x_{i}^{2}\right)^2\right]
        \end{equation*}
        kde $\vector{x}=(x_1,x_2,\dotsc,x_D)$. Počítejte pro $a=1, b=100, D=7$.
        Řešení hledejte v oblasti, kde jsou všechny souřadnice kladné, tj. kde $x_{i}\geq0$.

    \item Minimum nalezneme tím přesněji, čímje kratší bude krok $d$ v náhodné procházce.
        Pokud však krátký krok použijeme od samého začátku náhodné procházky, výpočet bude trvat velmi dlouho.
        Naprogramujte tedy minimalizační proceduru s \emph{proměnným krokem}: Začněte náhodnou procházku s delším krokem a krok postupně vhodně zmenšujte.
        Při optimálním naprogramování bude přesnost nalezení minima úměrná velikosti kroku na konci náhodné procházky.
\end{enumerate}
\end{document}