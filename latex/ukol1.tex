\documentclass[a4paper,11pt,twoside]{article}
\usepackage[utf8]{inputenc}	% Text coding
\usepackage[T1]{fontenc}
\usepackage{lmodern}
\usepackage[czech]{babel}
\usepackage{epsfig}
\usepackage{amsfonts,amsmath,amssymb}
\usepackage{graphicx}
\usepackage[unicode]{hyperref}
\usepackage{indentfirst}
\usepackage{fancyhdr}
\usepackage{xifthen}
\usepackage{amsthm,thmtools}
\usepackage{bold-extra}
\usepackage[dvipsnames]{xcolor}
\usepackage[subrefformat=simple,labelformat=simple]{subcaption} % Instead of subfigure
\usepackage{listings}
\usepackage{comment}
\usepackage{titlesec}
\usepackage{underscore}
\usepackage{makecell}       % Šířky čar v tabulkách

% Page size
\addtolength{\topmargin}{-1.5cm} %\addtolength{\textheight}{-10cm}
\addtolength{\textwidth}{4cm} \addtolength{\textheight}{4cm} % Width and height of the text
\addtolength{\voffset}{-0.5cm} % Top margin
\addtolength{\hoffset}{-2cm}
\setlength{\headheight}{15pt}

\DeclareMathOperator{\e}{e}

\def\vector#1{\boldsymbol{#1}}								% Vector
\renewcommand{\d}{\mathrm{d}}
\newcommand{\derivative}[3][]{\ifthenelse{\isempty{#1}}	    % Normal derivative
	{\frac{\d{#2}}{\d{#3}}}
	{\frac{\d^{#1}{#2}}{\d{#3}^{#1}}}
}
\newcommand{\im}{\mathrm{i}}

\def\makematrix#1{\begin{pmatrix}#1\end{pmatrix}}       % Matrix
\def\abs#1{\left|#1\right|}
\def\probability#1{\mathrm{Pr}\left[#1\right]}
\def\expectation#1{\mathrm{E}\left[#1\right]}
\def\dispersion#1{\sigma_{#1}^{2}}

\def\code#1{\textnormal{\texttt{#1}}}
\def\file#1{\textnormal{\textbf{\texttt{#1}}}}
\def\ghfile#1#2{\textnormal{\textbf{\texttt{\href{https://github.com/PavelStransky/PCInPhysics2021/blob/main/#1#2}{#2}}}}}

\def\abbreviation#1{\textnormal{\textsc{#1}}}

\begin{document}

\section*{Domácí úkol na 17.3.2022}
\subsection*{Runge-Kuttův algoritmus pro Lorenzův systém}
Naprogramujte a odlaďte řešení nelineární soustavy tří diferenciálních rovnic pro jednoduchý Lorenzův model vedení tepla v atmosféře
\begin{align}
    \derivative{x}{t}&=\sigma(y-x),\nonumber\\
    \derivative{y}{t}&=x(\rho-z)-y,\nonumber\\
    \derivative{z}{t}&=xy-\beta z\nonumber
\end{align}
s hodnotami parametrů $\sigma=10$, $\rho=28$ a $\beta=8/3$, počátečními podmínkami $(x_0, y_0, z_0)=(1,1,1)$ (na počátečních podmínkách zase tolik nezáleží), s krokem $\Delta t=0.01$ a na časovém intervalu $t\in\langle0,100\rangle$.
Vykreslete graf $z(x)$, případně (nepovinně) vykreslete 3D graf funkce $z(x,y)$.

K řešení diferenciálních rovnic použijte Runge-Kuttovu metodu 4. řádu:
\begin{align*}
    \vector{k}_{1}&=\vector{f}(\vector{w}_{i},t_{i}),\\
    \vector{k}_{2}&=\vector{f}\left(\vector{w}_{i}+\vector{k}_{1}\frac{\Delta t}{2},t+\frac{\Delta t}{2}\right),\\
    \vector{k}_{3}&=\vector{f}\left(\vector{w}_{i}+\vector{k}_{2}\frac{\Delta t}{2},t+\frac{\Delta t}{2}\right),\\
    \vector{k}_{4}&=\vector{f}\left(\vector{w}_{i}+\vector{k}_{3}\Delta t,t+\Delta t\right),\\
    \vector{\phi}_{i}&=\frac{1}{6}\left(\vector{k}_{1}+2\vector{k}_{2}+2\vector{k}_{3}+\vector{k}_{4}\right),\\
\end{align*}
kde $\vector{w}=(x,y,z)$, $\vector{f}$ je pravá strana soustavy rovnic Lorenzova systému a integrační krok se dělá stejně jako u ostatních procvičovaných jednokrokových algoritmů,
\begin{equation*}
    \vector{w}_{i+1}=\vector{w}_{i}+\vector{\phi}_{i}\Delta t.
\end{equation*}

Výsledná křivka je slavný \href{https://cs.wikipedia.org/wiki/Lorenz%C5%AFv_atraktor}{Lorenzův podivný atraktor} ve tvaru motýlích křídel, krerý zpopularizoval teorii klasického chaosu.

\end{document}