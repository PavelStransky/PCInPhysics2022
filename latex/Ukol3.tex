\documentclass[a4paper,11pt,twoside]{article}
\usepackage[utf8]{inputenc}	% Text coding
\usepackage[T1]{fontenc}
\usepackage{lmodern}
\usepackage[czech]{babel}
\usepackage{epsfig}
\usepackage{amsfonts,amsmath,amssymb}
\usepackage{graphicx}
\usepackage[unicode]{hyperref}
\usepackage{indentfirst}
\usepackage{fancyhdr}
\usepackage{xifthen}
\usepackage{amsthm,thmtools}
\usepackage{bold-extra}
\usepackage[dvipsnames]{xcolor}
\usepackage[subrefformat=simple,labelformat=simple]{subcaption} % Instead of subfigure
\usepackage{listings}
\usepackage{comment}
\usepackage{titlesec}
\usepackage{underscore}
\usepackage{makecell}       % Šířky čar v tabulkách

% Page size
\addtolength{\topmargin}{-1.5cm} %\addtolength{\textheight}{-10cm}
\addtolength{\textwidth}{4cm} \addtolength{\textheight}{4cm} % Width and height of the text
\addtolength{\voffset}{-0.5cm} % Top margin
\addtolength{\hoffset}{-2cm}
\setlength{\headheight}{15pt}

\DeclareMathOperator{\e}{e}

\def\vector#1{\boldsymbol{#1}}								% Vector
\renewcommand{\d}{\mathrm{d}}
\newcommand{\derivative}[3][]{\ifthenelse{\isempty{#1}}	    % Normal derivative
	{\frac{\d{#2}}{\d{#3}}}
	{\frac{\d^{#1}{#2}}{\d{#3}^{#1}}}
}
\newcommand{\im}{\mathrm{i}}

\def\makematrix#1{\begin{pmatrix}#1\end{pmatrix}}       % Matrix
\def\abs#1{\left|#1\right|}
\def\probability#1{\mathrm{Pr}\left[#1\right]}
\def\expectation#1{\mathrm{E}\left[#1\right]}
\def\dispersion#1{\sigma_{#1}^{2}}

\def\code#1{\textnormal{\texttt{#1}}}
\def\file#1{\textnormal{\textbf{\texttt{#1}}}}
\def\ghfile#1#2{\textnormal{\textbf{\texttt{\href{https://github.com/PavelStransky/PCInPhysics2021/blob/main/#1#2}{#2}}}}}

\def\abbreviation#1{\textnormal{\textsc{#1}}}

\begin{document}

\section*{Domácí úkol na 14.4.2022}
\subsection*{Hrátky s náhodnými čísly}
\begin{enumerate}
    \item
        Odlaďte svůj vlastní program na výpočet histogramu a jeho vykreslení do grafu.

    \item
        Nagenerujte čísla vybraná z Gaussovského normálního rozdělení (se střední hodnotou 0 a rozptylem 1) pomocí metody hit-and-miss a pomocí centrální limitní věty.
        Nakreslete a porovnejte histogramy.

    \item
        Vytvořte dva generátory čísel z rozdělení daného distribuční funkcí
        \begin{equation*}
            F(x)=\frac{1}{2}\left(1+\frac{2}{\pi}\arctan{x}\right).
        \end{equation*}
        V prvním generátoru využijte přímo distribuční funkci, pro druhý odvoďte hustotu pravděpodobnosti a použijte hit-and-miss metodu.
        Nakreslete a porovnejte histogramy.

    \item
        Vytvořte si vzdálený repozitář, nahrajte do něj svá řešení a umožněte mi do repozitáře přístup.
        Tím budu brát úkol jako odevzdaný.
\end{enumerate}

\end{document}